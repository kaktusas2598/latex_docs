\documentclass[8pt]{extarticle}

\pagestyle{empty} %No footer or header

\usepackage[utf8]{inputenc}
\usepackage{multicol}
\usepackage{color}
\usepackage{vmargin}
\usepackage{graphicx}
% Don't print section numbers
\setcounter{secnumdepth}{0}

\setlength{\columnsep}{0.5cm}
\setlength{\columnseprule}{2pt}
\def\columnseprulecolor{\color{black}}


\setlength{\parindent}{0pt}

\setmarginsrb%
{1.5cm}  % left
{0.5cm}  % top
{1.5cm}  % right
{1.0cm}  % bottom
{0ex}    % header height
{0ex}    % header separation
{0ex}    % footer height
{0ex}    % footser separation

\begin{document}

\title{Linux Command Line Notes}
\author{Nerijus Vilčinskas}
\date{2015-08-07}
\maketitle

\begin{multicols}{2}

%\begin{mdframed}
\section{INTRODUCTION}

\begin{tabular}{ll}
	\texttt{cal} & calendar \\
	\texttt{df} & free space on disk drives \\
	\texttt{free} & amount of free memory \\
	\texttt{exit} & exit terminal \\
\end{tabular}

\texttt{Ctrl+Alt+(F1-F6)} - access virtual terminals

\texttt{Alt+(F1-F6)} - switch virtual terminals

\texttt{Alt+F7} - come back to GUI
%\end{mdframed}

%\begin{mdframed}
\section{NAVIGATION}

\begin{tabular}{ll}
\texttt{pwd} & display current working directory\\
\texttt{ls} & list directory contents\\
\texttt{cd} \textit{dir} & change to directory \textit{dir} \\
\texttt{cd -} & change to previous working directory\\
\texttt{cd} & change to home dir
\end{tabular}

%\end{mdframed}

%\begin{mdframed}
\section{EXPLORATION}

\begin{tabular}{ll}
\texttt{ls} & list directory contents\\ 
& \texttt{-a} - all files\\
& \texttt{-d} - view directory\\
& \texttt{-F} - append file type indicator\\
& \texttt{-h} - in long format display file sizes nicely\\
& \texttt{-l} - long format\\
& \texttt{-r} - reversed order(normaly asc alphanumeric order)\\
& \texttt{-S} - sort by file size\\
& \texttt{-t} - sort by modification time\\		
\texttt{file} & view file type \\
\texttt{less} & view file contents
\end{tabular}
%\end{mdframed}

\begin{tabular}{ll}
Keybinding & \texttt{less} Commands \\ \hline
PAGE UP or b & next page \\
PAGE DOWN or Spacebar & previous page \\
Up Arrow or k & scroll line up \\
Down Arrow or j & scroll line down \\
G & move to EOF \\
1G or g & move to begginning of file \\
/characters & search \\
n & search of next occurance in search \\
h & help screen \\
q & quit less \\
\end{tabular}

%Maybe includegraphics here?
%Need to find better graphics
\subsection{Linux File Structure}
\textbf{/} - root dir, start of filesystem;
\textbf{/bin} - system binaries; \\
\textbf{/boot} - contains linux kernel, initial RAM disk image, bootloader;\\
\textbf{/dev} - device nodes;\\
\textbf{/etc} - system-wide configs and services start up scripts;\\
\textbf{/home} - writable directory for each ordinary user;\\
\textbf{/lib} - shared libraries used by core system;\\
\textbf{/lost+found} - used in case of filesystem corrupt for partial recovery;\\
\textbf{/media} - mount points for removable media mounted automatically (on modern Linux);\\
\textbf{/mnt} - mount points for removable devices mounted manually(older Linux);\\
\textbf{/opt} - for "optional" soft(mostly commercial);\\
\textbf{/proc} - virtual filesystem maintained by kernel, "files" are peppholes into kernel;\\
\textbf{/root} - root account home dir;\\
\textbf{/sbin} "system" binaries, preforming vital system tasks;\\
\textbf{/tmp} - temp files created by various programs;\\
\textbf{/usr} - likely largest dir, contains programs and support files used by regular users;\\
\textbf{/usr/bin} - binaries installed by distro;\\
\textbf{/usr/lib} -shared libraries used by programs in /usr/bin;\\
\textbf{/usr/local} - programs not included by distro, but used system wide;\\
\textbf{/usr/sbin} - more sysstem administration programs;\\
\textbf{/usr/share} - shared data betwen /usr/bin(default configs,icons,sound..);\\
\textbf{/usr/share/doc} - documentation files from packages;\\
\textbf{/var} - ever changing dir, includes databases, spoof files,mail;\\
\textbf{/var/log} - log file,s recordsof various system activities;\\

\section{MANIPULATING FILES AND DIRECTORIES}

\begin{tabular}{ll}
\texttt{cp} & copy files and directories\\
& \texttt{-a} - copy with attributes, including permissions \\
& and ownership\\
& \texttt{-i} - asks user for confirmation before overwriting file\\
& \texttt{-r} - copy directories and it's contents, used for dir copying\\
& \texttt{-u} - copy either non existant or newer versions file\\ 
& to destination dir\\
& \texttt{-v} - display informative messages while copying\\
\texttt{mv} & move/rename files and dirs\\
& \texttt{-i} - asks user for confirmation before overwriting file\\
& \texttt{-u} - move either non existant or newer versions file\\
& to destination dir\\
& \texttt{-v} - display informative messages while moving\\ \texttt{mkdir} & create dirs\\ \texttt{rm} & remove files and dirs\\
& \texttt{-i} -asks for confirmation before deleting existing file\\
& \texttt{-r} - recursively delete directories\\
& \texttt{-f} - ignore nonexistant files and\\
& do not prompt, overrides -i\\
& \texttt{-v} - display infromative messages\\
\texttt{ln} \textit{file link} & create hard link of \textit{file}\\
\texttt{ln -s} \textit{file link} & create symbolic link of \textit{file}\\
\end{tabular}

\subsection{Wilcards(Globbing)}
\begin{tabular}{ll}
* & Any character\\
? & Any single character\\
{[characters]} & Any character in set of characters\\
{[!characters]} & Any character not in set of characeters\\
{[[:class:]]} & Any character that belongs to specified class\\
\end{tabular}

\subsection{Character classes}
\begin{tabular}{ll}
{[:alnum:]}& Any alphanumeric character\\
{[:alpha:]} & Any alphabetic character\\
{[:digit:]} & Any numeral\\
{[:lower:]} & Any lowercase character\\
{[:upper:]} & Any uppercase character\\
\end{tabular}

\textbf{Wildcards} work in some GUI file managers too!

\subsection{Hard links}
\begin{itemize}
	\item Every file has a single has link to give file it's name
	\item Hard link cannot reference file outside its own filesystem
	\item Hard link cannot reference directory
\end{itemize}

\texttt{ls -i} to view files with their inode number for recognising same files created with hard links.

\subsection{Symbolic links}
\begin{itemize}
	\item Overcome hard links limitations
	\item Then link deleted, file is not deleted, if vice versa - link is broken
	\item File and it's link are indistinguishable
\end{itemize}

\section{WORKING WITH COMMANDS}

\begin{tabular}{ll}
	\texttt{type} & show command name is interpreted\\
	\texttt{which} & show which executable is going to be launched\\
	\texttt{man} & show command manual page\\
	\texttt{apropos(man -k)} & display list of appropriate commands\\
	\texttt{man } \textit{section} & display man entry for section \textit{section}\\
	\texttt{info} & display command's info entry\\
	\texttt{whatis} & very brief description of command\\
	\texttt{alias} & create command alias\\
	\texttt{unalias} & unbind alias\\
	\texttt{help} & help for shell builtins
\end{tabular}

\subsection{Command types}
\begin{itemize}
	\item Executables - either compiled or scripted
	\item Shell builtin commands
	\item Shell Functions
	\item Aliases
\end{itemize}

\texttt{--help} - many executables have this option to display command syntax and options.

\texttt{man} uses \texttt{less} pager most of the times, so \texttt{less} shortcuts will work here!

\subsection{man sections}
\begin{tabular}{ll}
1 &  User commands\\
2 & Programming interfaces for kernel system calls\\
3 & Programming interfaces to the C library\\
4 & Special files such as devices nodes and drivers\\
5 & File formats\\
6 & Games and amusements such as screensavers\\
7 & Miscellaneous\\
8 & System administration commands
\end{tabular}

\subsection{info commands}
\begin{tabular}{ll}
? &	Display command help\\
PAGE UP or BACKSPACE & Display previous page\\
PAGE DOWN or Spacebar & Display next page\\
n &	Display next node\\
p &	Display previous node\\
u &	Up - Display parent node of \\ &currently displayed note(menu)\\
ENTER &	Follow hyperlink at cursor pos\\
q &	Quit.
\end{tabular}

Most other documentation is in \textbf{/usr/share/doc!!!}

\texttt{alias} \textit{name = 'string'}

Seperate shell comands with \textbf{;}

\section{REDIRECTION}

\begin{tabular}{ll}
\texttt{cat} & concatenate files,display, join files, read from stdin\\
\texttt{sort} & sort lines of text\\
\texttt{uniq} & report or omit repeated lines from sorted stdin\\
& \texttt{-d} show repated lines(default action is ommit them)\\
\texttt{wc} & print newline, word and byte counts for each file\\
& \texttt{-l} - show only newline count\\
\texttt{grep} & print lines matching pattern\\
& \texttt{-i} -ignore case\\
& \texttt{-v} - print lines that do not match pattern\\
\texttt{head} & output first part of file\\
& \texttt{-n} \textit{number} - specify how many lines to print(default-10)\\
\texttt{tail} & output last part of file\\
& \texttt{-n} \textit{number} - specify how many lines to print(default-10)\\
& \texttt{-f} - real time file end view\\
\texttt{tee} & read from standard input \& write to standard output \\
& and files
\end{tabular}

\begin{tabular}{ll}
	\texttt{>} & redirect stdout with truncating\\
	\texttt{>>} & redirect stdout with appending\\
	\texttt{2>} & redirect to stderr(2 is file descriptor, but it wont be always 2)\\
	\texttt{\&>} & redirect both stdout and stderr\\
	%\textgreater - >
	& alternate old school way: command \textgreater out\_file 3\textgreater\&1
\end{tabular}

\subsubsection{File descriptors}
\begin{tabular}{ll}
	\texttt{1} & stdin\\
	\texttt{2} & stdout\\
	\texttt{3} & stder\\
\end{tabular}

\texttt{cat} \textit{movie.mpeg.0}* \textgreater  \textit{movie.mpeg} - \texttt{cat} is good for joining files!

$|$ (pipeline) - used for redirecting one command stdout to stdin of another command, ex.: \texttt{ls /usr/bin | less}

\texttt{ls /usr/bin | tee ls.txt | grep nu} - \texttt{tee} outputs \texttt{ls} to file and sends \texttt{ls} for pattern matching.

\section{SEEING WORLD AS SHELL(EXPANSION, QUOTING, ESCAPING)}

\begin{tabular}{ll}
	\texttt{echo} - display line of text
\end{tabular}

\subsection{Expansions}

\subsubsection{Pathname}

\texttt{ls *}, \texttt{echo *S}, \texttt{echo [[ :upper:]]*} - shell expands wildcards.

\texttt{echo .[!.]?*} - hidden files pathname expansion.

\subsubsection{Tilde}

\textasciitilde expands into home dir, \texttt{\textasciitilde username} for other users home dirs.

\subsubsection{Arithmetic}

\texttt{+,-,/,*,\%,**} - supports only integer division.

\texttt{echo \$(( \$(( 5**2)) * 3))} - $5^{2}*3$

\texttt{echo \$(((5**2) * 3))} – same, single parentheses for grouping.

\texttt{echo \$(( 5/2))} – $2$

\subsubsection{Brace}

\texttt{echo 1\{A,B,C\}9} – 1A9 2A9 3A9.

\texttt{echo No\{1..5\}} – No1 No2 No3 No4 No5(May also be used with chars and reverse order).

\texttt{echo 1\{A\{1,2\},B\{3,4\}\}2} – 1A12 1A22 1B33 1B42 – nesting.

Good for making lists of files or directories!

\subsubsection{Parameter}

\texttt{echo \$USER} – expands into user variable(parameter).

Useful in shell scripts!

\subsection{Command Substitution}

\texttt{echo \$(ls)} – expands command output.

\texttt{file \$(ls /usr/bin/* | grep zip)} – entire pipeline.

\texttt{ls -l `  which cp\`} - alternate old school syntax.

\subsection{Quoting}

\textbf{Double quotes} – 
%Missing $ inserted ERROR
%special chars lose meaning(exceptions are \texttt{\$} ,\backslash, `). So paramether, arithmetic expansions and command substition are still allowed.

Examples: \texttt{cat “Bad filename”} ; \texttt{echo “Space works!”} ; \texttt{echo “\$(cal)”} - nice calendar(show newlines).

\textbf{Single quotes} – supresses ALL expansions.

%Escaping characters – use “\backslash” to quote single characters. \$, \backslash.. And also lets to put special character into file name – \texttt{touch \&stupid}

%Escape sequences - \backslash a, \backslash b, \backslash n, \backslash r, \backslash t (use echo -e to enable interpretation).

\section{KEYBOARD TRICKS}

\begin{tabular}{ll}
	\texttt{clear} & clear screen\\
	\texttt{history} & display history list\\
\end{tabular}

Command line editing is implemented for bash by \textbf{Readline}.

\subsection{Cursor Movement Commands}
\begin{tabular}{ll}
	CTRL + A & Cursor at the beginning of line\\
	CTRL + E & Cursor at the end of line\\
	ALT + F & Move cursor forward one word\\
	ALT + B & Move cursor backward one word\\
	CTRL + L & Clear the screen
\end{tabular}

\subsection{Text Editing Commands}
\begin{tabular}{ll}
	CTRL + D & Delete character at cursor position\\
	CTRL + T & Transpose character at cursor with one preceeding it\\
	ALT + T & Transpose word at cursor with one preceeding it\\
	ALT + L & Convert characters from cursor to the end of line to lowecase\\
	ALT + U & Convert character from cursor to the end of line to uppercase
\end{tabular}

\subsection{Cut and Paste Commands}
\begin{tabular}{ll}
	CTRL + K & Kill text from cursor to the end of line\\
	CTRL + U & Kill text from cursor to the beginning of line\\
	ALT + D & Kill text from cursor to the end of word\\
	ALT + BACKSPACE & Kill Text from cursor to beginning of line. If cursor at begin, kill previous word\\
	CTRL + Y & Yank text from kill-ring and insert at cursor position
\end{tabular}

\textbf{meta} key is usually \textbf{ALT}!

\textbf{Completion} - is mechanism that on pressing \textbf{TAB} completes commands, file names, hosts, usernames, variables.

\begin{tabular}{ll}
	ALT + ? & display list of possible completions, same as pressing TAB second time.\\
	ALT + * & insert all possible completions.
\end{tabular}

It is also possible to program completions. see \texttt{set | less}

For more Info on Command Line Editing and Completion read \texttt{man bash} \textbf{READLINE} section!

\subsection{Using History}

History is saved in \textbf{.bash\_history} file in home dir.

\begin{tabular}{ll}
	\texttt{!88} & performs history expansion(executes 88 item in history list)
\end{tabular}

\subsection{History Commands}
\begin{tabular}{ll}
	CTRL + P & Move to previous history entry\\
	CTRL + N & Move to next history entry\\
	ALT + \textless & Move to beginning of history list\\
	ALT + \textgreater & Move to end of history list\\
	CTRL + R & Reverse incremental search\\
	ALT + P & Reverse search, non-incremental. Hit Enter \\
	& before actually performing search\\
	ALT + N & Forward search, non-incremental\\
	CTRL + O & Execute current history entry and move to text\\
	CTRL + J & Copy current history entry to terminal
\end{tabular}

\subsection{History Expansion Commands}
\begin{tabular}{ll}
	\textit{!!} & Repeat last command\\
	\textit{!number} & Repeat history entry item \textit{number}\\
	\textit{!string} & Repeat last history entry starting with \textit{string}\\
	\textit{!?string} & Repeat last history entry containing \textit{string}
\end{tabular}

\begin{tabular}{ll}
	\texttt{script} \textit{[file]} & record entire shell session into file 
\end{tabular}

\section{PERMISSIONS}

\begin{tabular}{ll}
	\texttt{id} & display user's identity\\
	\texttt{chmod} & change file's mode\\
	\texttt{umask} & set default file permissions\\
	\texttt{su} & run shell as another user\\
	& \texttt{-} or \texttt{-l} for login shell(user's environment\\
	& is loaded and set to user's home dir)\\
	& may specify user(defaults to superuser)\\
	\texttt{sudo} & execute command as another user\\
	\texttt{chown} \textit{[owner][:[group]] file\ldots} & change file's owner\\
	\texttt{chown} \textit{user:} & changes owner to user and group \\
	& to user's login group\\
	\texttt{chown} \textit{:users} & same owner, group to users\\
	\texttt{chgrp} & change file's group ownership\\
	\texttt{passwd} & change user's password
\end{tabular}

\subsection{File attributes}

\texttt{-rwxrw-r-- }
\texttt{0\ldots\ldots\ldots9}
\begin{tabular}{ll}
	\texttt{0} & file type\\
	\texttt{1-3} & owner permissions\\
	\texttt{4-6} & group permissions\\
	\texttt{7-9} & world permissions
\end{tabular}

\subsection{File types}

\begin{tabular}{ll}
	\texttt{-} & regular file\\
	\texttt{d} & directory\\
	\texttt{l} & symbolic link(holds dummy attributes)\\
	\texttt{c} & character special file(device that handles data as streams of bytes, such as terminal or modem)\\
	\texttt{b} & block special file(device that handles data in blocks, such as CDROM or hard drive)
\end{tabular}

\subsection{File permissions}

\end{multicols}
\end{document}
